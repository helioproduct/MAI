\documentclass[a4paper,12pt]{article}

\usepackage[T2A]{fontenc}
\usepackage[utf8]{inputenc}

\usepackage[english,russian]{babel}
\usepackage{amsmath,amsfonts,amssymb,amsthm,mathtools}

\author{Happy Helio}
\title{Общие принципы и математика в \LaTeX{}}
\date{\today}

\begin{document}
\maketitle
\newpage

\section{Обычный текст}

Наша первая строка \\ 
Важное можо выделить \textbf{жирным} \\
\textit{Строка, написанная курсивом} \\
\underline{Подчеркнутный текст}
\fbox{текст в рамке} \\
Дефис -- не тире \\
Дефис -  не тире \\
<<Кавычки>>

\section{Мир формул}

Наша первая формула $100+100=200$, ага.
\[100+100=200\] %формула по середине

\begin{equation}\label{pifagor}
a^2+b^2=c^2
\end{equation}

Теорема Пифагора \eqref{pifagor} изучается в 8-ом классе. 
Она упоминается на странице \pageref{pifagor}

\subsection{Дроби} 

% Для того чтобы убрать нумерацию разделов, подразделов
% необходимо поставить *
% Пример:  \subsection*{Рациональные дроби}

$\frac{1}{3}+\frac{1}{3}=\frac{2}{3}$ Вот вам и дроби. Выглядит не очень.

\[\frac{1}{3}+\frac{1}{3}=\frac{2}{3}\]
А так намного лучше.

\subsection{Скобки}

\[ (2+3)\times 5=25\]

\[ (\frac{4}{2}+3)\times 5=25 \]

\[ \left[\frac{4}{2}+3\right]\times 5 =25 \]

\[ \left(\frac{4}{2}+3\right)\times 5 =25 \]

\[ \left\{\frac{4}{2}+3\right\}\times 5 =25 \]


\subsection{Индексы}

Грузы массой $m_1$, $m_{12}$ \\
Тоже самое:

\[ m_1, m_{12} \]

\subsection{Стандартные формулы}

\[ \sin x=0 \]
\[ \arctan x=\sqrt[5]{3} \]
\[ \log_{x-1}{(x^2-4x+5)} \geqslant 2\]
\[ \lg_{x-1}{(x^2-4x+5)} \geqslant 2\]
\[ \ln_{x-1}{(x^2-4x+5)} \geqslant 2\]

\subsection{Не очень стандартные формулы}

\[\sum_{i=1}^{n}a_i \times b_i \]

\[I = \int r^2dm \]

\[I = \int_{0}^{1} r^2dm \]

\[I = \int\limits_{0}^{1} r^2dm \]

\subsection{Символы}

\[2\times 2\neq 5 \]

\[x \cap y \]



\end{document}
