\documentclass[a4paper,12pt]{article}

\usepackage[T2A]{fontenc}
\usepackage[utf8]{inputenc}

\usepackage[english,russian]{babel}
\usepackage{amsmath,amsfonts,amssymb,amsthm,mathtools}
\usepackage{graphicx}

\title{\textbf{Отчет по лабораторной работе № 21}}
\author{Студент группы М80-108Б-21 Попов Николай Александрович}
\date{22 марта 2022 г.}

\begin{document}
\maketitle

\begin{tabular}{|p{4cm}|p{8cm}|}
\hline
Студент & Попов Н. А. \\
\hline
Группа & М80-108Б-21 \\
\hline
Преподаватель & Трубченко Н. М. \\ 
\hline
Вариант & 1 \\
\hline
Дата & 10.03.22 \\
\hline
\end{tabular}

\newpage

\section*{Цель работы}
Составить программу выполнения заданных действий над файлами на одном
из интерпретируемых командных языков OC UNIX \\ (Shell, Cshell, Bash, ...) 

\section*{Задание}
Копирование всех файлов, для которых установлена защита от записи от всех
пользователей. Имена копий герерировать путем добавления указанного префикса.
Файлы-копии должны быть доступны для записи всем пользователям.

\section*{Оборудование (студента)}
Процессор AMD Ryzen 7 5700U @ 4.30 GHz с ОП 16 Гб

\section*{Программное обеспечение (студента)}
Операционная система семейства Linux, наименование Manjaro Linux,
интерпретатор команд Bash версии 5.0

\section*{Идея, метод, алогоритм решения задачи}
% Usage: ./copy_wt.sh --src=[] --dst=[] --prefix=[]

Вывод команды \textit{ls --l} в директории \textit{--src} передаем
команде \textit{grep} и фильтруем ее вывод при помощи регулярного выражения.
Из полученного результата формируем массив имен файлов. Создаем новые файлы в директории \textit{--dst} с указанным префиксом, записываем в них содержимое
соотвествующих файлов из директории \textit{--src} \\

\newpage
\section*{Распечатка протокола}

\begin{figure}[h]
\centering
\includegraphics[scale=0.2]{screenshot1.png}
\caption{Содержимое директориии \textit{--src}}
\end{figure}

\begin{figure}[h]
\centering
\includegraphics[scale=0.2]{screenshot2.png}
\caption{Использование скрипта: \textit{./copy.sh --src=[] --dst=[] --prefix=[]}}
\end{figure}

\begin{figure}[h]
\centering
\includegraphics[scale=0.2]{screenshot3.png}
\caption{Содержимое \textit{--dst} после применения скрипта к директории \textit{--src}}
\end{figure}

\section*{Вывод}
Данная работа впервые познакомила меня с интерпретируемым языком программирования Bash.
Я научился использовать его для написания небольших программ, работающих внутри OC UNIX. 

\end{document}
